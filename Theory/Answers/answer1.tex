\subanswer[(1)]{20}
\subsubanswer[(1.1)]{10}
质点的运动方程为:
\begin{cphosequation}[2]
m\ddot{x} = -kx
\end{cphosequation}
\begin{cphosequation}[2]
m\ddot{y} = -ky
\end{cphosequation}
解得:
\begin{cphosequation}[3]
x(t) = A\cos(\omega t + \phi_x)
\end{cphosequation}
\begin{cphosequation}[3]
y(t) = B\sin(\omega t + \phi_y)
\end{cphosequation}
其中 $\omega = \sqrt{k/m}$。根据初始条件,最终轨迹方程为椭圆。

\subsubanswer[(1.2)]{10}
角动量 $L = \vec{r} \times \vec{p}$。
\begin{cphosequation}[5]
\frac{d\vec{L}}{dt} = \frac{d\vec{r}}{dt} \times \vec{p} + \vec{r} \times \frac{d\vec{p}}{dt} = \vec{v} \times (m\vec{v}) + \vec{r} \times \vec{F}
\end{cphosequation}
因为 $\vec{v}$ 和 $m\vec{v}$ 平行,$\vec{r}$ 和 $\vec{F}$ 平行(中心力),所以
\begin{cphosequation}[5]
\frac{d\vec{L}}{dt} = 0
\end{cphosequation}
角动量守恒。

\subanswer[(2)]{20}
从能量守恒方程
\begin{cphosequation}[10]
E = \frac{1}{2}m v^2 - \frac{k}{r}
\end{cphosequation}
可以直接解出速度 $v$:
\begin{cphosequation}[10]
v = \sqrt{\frac{2}{m}\left(E + \frac{k}{r}\right)}
\end{cphosequation}
这里 $E$ 是系统的总能量。
\begin{figure}[H]
\centering
\includegraphics[width=0.4\textwidth]{../Assets/Murasame2.jpg}
\cphoscaption{丛雨太可爱了吧}
\label{fig:ans1}
\end{figure}
